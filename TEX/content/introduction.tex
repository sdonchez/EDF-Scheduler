% !TeX root = ../SDonchezResearchIReport.tex

\section{Introduction} \label{sec:introduction}

\IEEEPARstart{F}{ield} Programmable Gate Arrays, or FPGAs, have long been utilized in devices benefiting from hardware acceleration of processes unsuitable for execution on a traditional processor, but which don't warrant the time consuming and cost-prohibitive process of ASIC design. Accordingly, it should come as no surprise that, as much of the world pivots from on-site datacenters and computing resources to hybrid or cloud based platforms, Cloud Service Providers (CSPs) have shown an interest in providing FPGA-as-a-service resources to their tenants. Amazon, Google, and Microsoft all offer such services as part of their respective cloud platforms, as do many of the other major players in the industry. 

Currently, these offerings mirror the traditional FPGA implementation - a single physical device per tenant instance. This is due to a variety of factors, mostly centered around device security for both the tenant and the CSP. However, research is underway that seeks to utilize Dynamic Partial Reconfiguration (DPR) as a means to enable CSPs to partition large FPGAs into multiple virtual devices, each of which can be allocated to a tenant. This would allow CSPs to utilize much more economical devices, driving down cost for tenants and likely spurring increased adoption. Before this technology can see widespread adoption, it must address a number of security concerns surrounding co-tenancy, as well as trust relationships between the CSP and the tenant.

This effort seeks to address the latter concern, that of CSP - tenant trust. Understandably, many tenants are hesitant to provide their Intellectual Property (IP) directly to the CSP, as there is financial incentive for a CSP to integrate some of that IP into their own accelerator offerings. Accordingly, tenants seek to upload encrypted IP to their partitions, with assurances that it is not possible for the CSP to directly access the decrypted content. This is feasible, and has already been researched in academia. However, the solution posed in the literature is inefficient in its duplication of components, and furthermore contains a flaw that entirely violates the integrity of the bitstream as it pertains to its inaccessibility in its decrypted form by the CSP. This research effort seeks to address that vulnerability, while also providing performance enhancements to the original design. Specifically, this work, as a part of the larger effort, seeks to provide an efficient way of allocating shared decryption resources in a manner that reduces overhead while still ensuring the integrity of the IP's encryption through all CSP-accessible parts of the tenancy cycle.

\subsection{Organization of this Work} \label{sec:OrgWork}
This work is organized into several sections. Section \ref{sec:Proposal} outlines the overarching proposal for the larger research effort of which this work is a part. This discussion of the larger effort is provided as context for the specific implementation described in later sections. Section \ref{sec:EDFAlgorithm} outlines the design of the scheduling algorithm for allocating the AES Decryption cores between the multiple tenant partitions on the FPGA. Section \ref{sec:Impl} describes the implementation of this algorithm, while Section \ref{sec:findings} describes the findings that have resulted from this implementation process. Section \ref{sec:conclusion} concludes the report.