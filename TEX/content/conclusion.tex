% !TeX root = ../SDonchezResearchIReport.tex

\section{Conclusion}\label{sec:conclusion}
In this paper, a novel solution for scheduling the allocation of shared IP cores on a multi-tenant FPGA is proposed and designed utilizing an Earliest Deadline First (EDF) based scheduling algorithm executing on a general purpose processor co-located with the programmable logic as part of a heterogenous embedded system. This paper also describes the implementation of this design, along with a stimulus application designed to execute it, both authored in C++ and utilizing both Windows and Linux InterProcess Communication functionality for flexibility of evaluation. The paper then discusses the results of the implementation, both with regards to primary objectives (the effectiveness of the algorithm itself), and secondary objectives (performance, etc.)

\subsection{Avenues for Future Work}\label{subsec:future}
This scheduler implementation outlines a solution for the scheduling of common IPs between multiple VFPGAs on a heterogenous system, but it is conceivable that a CSP may desire to implement VFPGAs on a purely PL device, such that no onboard processor exists. Although such an implementation would not facilitate the implementation of much of the larger system outlined in Section \ref{sec:Proposal} above, it does pose a number of avenues for future work specific to the implementation of a scheduling algorithm itself. Similarly, the current implementation's testbench is understandably primitive in nature compared to the larger, more complex stimulus applicationset that a CSP would utilize. Finally, it is worth noting that a logical route for expansion is the implementation of the larger system outlined in Section \ref{sec:Proposal} above, which would necessarily depend on the integration of the components developed herein.